\documentclass[12pt]{article} 
\usepackage[utf8]{inputenc}
\usepackage{geometry}
\geometry{letterpaper}
\usepackage{graphicx} 
\usepackage{parskip}
\usepackage{booktabs}
\usepackage{array} 
\usepackage{paralist} 
\usepackage{verbatim}
\usepackage{subfig}
\usepackage{fancyhdr}
\usepackage{sectsty}
\usepackage{enumitem}

\pagestyle{fancy}
\renewcommand{\headrulewidth}{0pt} 
\lhead{}\chead{}\rhead{}
\lfoot{}\cfoot{\thepage}\rfoot{}


%%% ToC (table of contents) APPEARANCE
\usepackage[nottoc,notlof,notlot]{tocbibind} 
\usepackage[titles,subfigure]{tocloft}
\renewcommand{\cftsecfont}{\rmfamily\mdseries\upshape}
\renewcommand{\cftsecpagefont}{\rmfamily\mdseries\upshape} %

\usepackage{amsmath}
\usepackage{amssymb}
\usepackage{mathtools}
\usepackage{empheq}
\usepackage{xcolor}
\usepackage{bbm}
\usepackage{tikz}
\usepackage{pgfplots}
\pgfplotsset{compat=1.18}

\newcommand{\ans}[1]{\boxed{\text{#1}}}
\newcommand{\vecs}[1]{\langle #1\rangle}
\renewcommand{\hat}[1]{\widehat{#1}}
\newcommand{\F}[1]{\mathcal{F}(#1)}
\renewcommand{\P}{\mathbb{P}}
\newcommand{\R}{\mathbb{R}}
\newcommand{\E}{\mathbb{E}}
\newcommand{\Z}{\mathbb{Z}}
\newcommand{\ind}{\mathbbm{1}}
\newcommand{\qed}{\quad \blacksquare}
\newcommand{\brak}[1]{\left\langle #1 \right\rangle}
\newcommand{\bra}[1]{\left\langle #1 \right\vert}
\newcommand{\ket}[1]{\left\vert #1 \right\rangle}
\newcommand{\Unif}{\text{Unif}\,}
\newcommand{\mfX}{\mathfrak{X}}

\title{APMA 1690: Quiz 1}
\author{Milan Capoor}
\date{27 October 2023}

\begin{document}
\maketitle

\section*{Problem 1}
\emph{Consider the following PDF}
\[p(t) = \frac{1}{2\sqrt t} \cdot \ind_{\{0 < t < 1\}}.\]
\emph{Compute the CDF $F(t)$ associated with the PDF $p(t)$, i.e. $F(t) = \int_{-\infty}^t p(x)\; dx$}

    \color{blue}
        \begin{align*}
            F(t) &= \int_{-\infty}^t p(x)\; dx\\
            &= \int_{-\infty}^t \frac{1}{2\sqrt x} \cdot \ind_{\{0 < x < 1\}} \; dx\\
            &= \int_{-\infty}^0 0 \; dx + \int_0^t \frac{1}{2\sqrt x} \; dx \\
            &= \int_0^t \frac{1}{2\sqrt x} \; dx\\
            &= \frac{1}{2}\left[2\sqrt x\right]_0^t\\
            &= \sqrt{t}
        \end{align*}
        So 
        \[\boxed{F(t) = \begin{cases}
            0 \qquad t \leq 0\\
            \sqrt{t} \quad\; 0 < t < 1\\
            1 \qquad t \geq 1
        \end{cases}}\]
    \color{black}
\pagebreak

\section*{Problem 2}
\emph{Let $U_1, U_2, \dots,\; U_n, \dots$ be $\R$-valued random variables defined on the probability space $(\Omega, \P)$, and $U_1, U_2, \dots,\; U_n, \dots \overset{iid}{\sim} \Unif(0, 1)$. Define random variables $V_n$ by}
    \[V_n(\omega) = \frac{1}{n}\sum_{i=1}^n (U_{2i-1}(\omega))^2 \cdot \ind_{\{U_{2i}(\omega) < 0.5\}}, \qquad n = 1, 2, \dots\]
\emph{What is the value $v$ such that $\P(\omega \in \Omega: \lim_{n\to\infty} V_n(\omega) = v) = 1$? Provide the value of $v$ and your justification.}

    \color{blue} 
        By the LLN, for large enough $n$, the sample average of a sequence $X_n$ converges to the expected value of $X_1$ with probability $1$ given that $X_n$ is a sequence of iid random variables and the expected value is defined. 

        Since $V_n$ is defined as the average of a transformation of the sequence $U_1, \dots, \; U_n$ which are iid RVs, $\{V_i\}_{i=1}^n$ is a sequence of iid RVs. Thus, to find $v$, we just need to determine the expected value of
        \[\tilde V(\omega) = U_1(\omega)^2 \cdot \ind_{\{U_2(\omega) < 0.5\}}\]

        First we need to find the PDF of the associated distribution:
        \begin{align*}
            F_{\tilde V}(t) &= \P(U_1^2 \cdot \ind_{U_2 < 0.5} \leq t)\\
            &= \begin{cases}
                0 \qquad t < 0\\
                \P(U_1^2 \cdot \ind_{U_2 < 0.5} \leq t) \qquad 0 \leq t \leq 1\\
                1 \qquad t > 1
            \end{cases}
        \end{align*}
        Looking at the middle case, 
        \[\P(U_1^2 \cdot \ind_{U_2 < 0.5} \leq t) = \P(U_1 \cdot \ind_{U_2 < 0.5} \leq \sqrt t)\]
        because the indicator function takes values only in $\{0, 1\}$ and is invariant under the square root. 

        We now consider all combinations of $U_1, U_2$ such that the product is less than or equal to $\sqrt t$ with $t \in [0, 1]$:
        \begin{enumerate}
            \item $\P(U_1 = 0) = 0$       
            \item $\P(U_2 \geq 0.5) = \frac{1}{2}$ 
            \item $\P((U_1 \leq \sqrt t) \cap (U_2 < 0.5)) = \sqrt t \cdot \frac{1}{2} = \frac{\sqrt t}{2}$
        \end{enumerate}
        We can sum these to get the total probability:
        \[F_{\tilde V} = \begin{cases}
            0 \qquad\qquad t < 0\\
            \frac{1 + \sqrt t}{2} \qquad\; 0 \leq t \leq 1\\
            1 \qquad\qquad t > 1
        \end{cases}\]
        and differentiate to at last get the PDF:
        \[p_{\tilde V} = \frac{1}{4\sqrt t}\]

        Finally, we can confirm the expected value exists:
        \begin{align*}
            \E[\tilde V] &= \int_{-\infty}^\infty \big\vert t^2 \cdot \ind_{t < 0.5} \big\vert \cdot \frac{1}{4\sqrt t} \; dt \\
                &= \int_{-\infty}^\infty \frac{1}{4}t^{\frac{3}{2}} \cdot \ind_{t < 0.5} \; dt \\
                &= \int_0^{0.5} \frac{1}{4}t^{\frac{3}{2}}\; dt\\
                &= \frac{1}{10}[t^{5/2}]_0^{0.5} = \frac{\sqrt{32}}{10} = \frac{2}{5}\sqrt 2 < \infty
        \end{align*}
        And in fact, because the expression inside the absolute value is always positive, the expected value is the value of the integral itself. Thus, 
        \[v = \E[\tilde V] = \boxed{\frac{2\sqrt 2}{5}}\]
    \color{black}
\pagebreak

\section*{Problem 3}
\emph{Let $X$ be a continuous random variable whose PDF is the following}
    \[p(x)=  cx^2 \cdot \ind_{\{0 \leq x \leq 1\}}\]
\emph{where c is some constant.}
\begin{enumerate}[label=(\alph*)]
    \item \emph{What is the value of $c$?}
        
        \color{blue}
            By the indicator function, $p(x)$ is only non-zero on the range $0 \leq x \leq 1$. Thus, the value of the CDF must be $1$ at $x=1$:
            \[F(1) = \int_0^1 cx^2 \; dx = \frac{c}{3} = 1 \implies \boxed{c = 3}\]
        \color{black}

    \item \emph{Find the CDF of $X$.}
    
        \color{blue}
            \begin{align*}
                F(t) &= \int_{-\infty}^t p(x)\; dx\\
                    &= \int_{-\infty}^t 3x^2 \cdot \ind_{\{0 \leq x \leq 1\}} \; dx\\
                    &= \int_0^t 3x^2\; dx = [x^3]_0^t = t^3
            \end{align*}
            So 
            \[\boxed{F(t) = \begin{cases}
                0 \qquad t < 0\\
                t^3 \quad \;\; 0 \leq t \leq 1\\
                1 \qquad t > 1
            \end{cases}}\]
        \color{black}

    \item \emph{Suppose $U$ is a $\R$-valued random variable defined on the probability space $(\Omega, \P)$ and $U \sim \Unif(0, 1)$. Construct a random variable $\tilde X$ using $U$ such that the CDF of $\tilde X$ is the CDF you got in (b). You have to present your answer in an analytic form, i.e., the infimum sign ``inf'' is forbidden in your answer.}
    
        \color{blue}
            By the inverse CDF method, 
            \[\tilde X(\omega) = G(U(\omega))\]
            where 
            \[G(u) = \inf\{t \in \R: F(t) \geq u\}, \qquad u \in (0, 1)\]
           
            Here,
            \[F(t) = t^3 \qquad (0 \leq t \leq 1)\]
            So 
            \begin{align*}
                G(u) &= \inf\{t \in \R: F(t) \geq u\}\\
                &= \inf\{t \in \R: t^3 \geq u\}\\
                &= \inf\{t \in \R: \sqrt[3]{u} \leq t\}\\
                &= \sqrt[3]{u}
            \end{align*}
            So 
            \[\boxed{\tilde X = \sqrt[3]{U}}\]
        \color{black}

    \item \emph{Justify your answer to (c).}
    
        \color{blue}
            \begin{align*}
                F_{\tilde X}(t) &= \P(\tilde X \leq t)\\
                &= \P(\sqrt[3]{U} \leq t)\\
                &= \P(U \leq t^3)\\
                &= F_U(t^3)\\
                &= \begin{cases}
                    0 \qquad \;t < 0\\
                    t^3 \qquad 0 \leq t \leq 1\\
                    1 \qquad \;1 < t 
                \end{cases}\qed
            \end{align*}
        \color{black}

\end{enumerate}
\pagebreak

\section*{Problem 4}
\emph{ Let $d$ be any positive integer. Recall/accept the following fact: The d-dimensional simple random walk is a homogeneous Markov chain whose state space is $\mfX = \Z^d = \Z \times \Z \times \dots \times \Z$.}

\emph{Prove the following: The d-dimensional simple random walk is irreducible.
(Hint: You may consider applying the following result that you have proved: if $A \subset B$, then $\P(A) \leq \P(B)$.)}

    \color{blue}
        A homogeneous Markov chain is irreducible if for all $x, y \in \mfX$, 
        \[\rho_{xy} = \P(T_y < \infty \; | \; X_0 = x) > 0\]

        The condition $T_y < \infty$ is equivalent to the condition that $X_n = y$ for some finite $n$. Thus, 
        \[\rho_{xy} = \P(X_n = y \; | \; X_0 = x)\]
    
        But we notice that the $1$-dimensional random walk is a subset of the $d$-dimensional walk: Consider states $x_{n}, x_{n+1} \in \Z^d$: 
        \[x_n = \begin{pmatrix}
            x_n^{(1)}\\
            x_n^{(2)}\\
            \vdots\\
            x_n^{(d)}
        \end{pmatrix}, \qquad x_{n+1} = \begin{pmatrix}
            x_{n+1}^{(1)}\\
            x_{n+1}^{(2)}\\
            \vdots\\
            x_{n+1}^{(d)}
        \end{pmatrix}\]

        Then $x_{n+1} = x_n + \xi_n$ and element wise, $x_{n+1}^{(1)} = x_{n}^{(1)} + \xi_n^{(1)}$ where $x_{n+1}^{(1)}, x_{n}^{(1)} \in \Z$ and $\xi_n^{(1)} \in \{1, -1\}$.

        Thus, we can say that 
        \[\P(X_{n+1}^{(1)} = x_{n+1}^{(1)} \; | \; X_n^{(1)} = x_n^{(1)}) \leq \P(X_{n+1} = x_{n+1} \; | \; X_n = x_n)\]
        because for any events $A \subset B \implies \P(A) \leq \P(B)$. 

        In HW 5 we proved the 1-d SRW is homogeneous, so $\exists p$ which is not a function of $n$ such that:  
        \[p(x_{n}^{(1)}, x_{n+1}^{(1)}) \leq \P(X_{n+1} = x_{n+1} \; | \; X_n = x_n)\]

        So calculating for the 1-d random walk,
        \begin{align*}
            p(y^{(1)}, x^{(1)}) &= \P(X_n^{(1)} = y^{(1)} \; | \; X_0^{(1)} = x^{(1)})\\
            &= \frac{\P(X_n^{(1)} = y^{(1)}, X_0^{(1)} = x^{(1)})}{\P(X_0^{(1)} = x^{(1)})}\\
            &= \frac{\P(X_n^{(1)} = y^{(1)}, X_{n-1}^{(1)} = x_{n-1}^{(1)}, \dots, \; X_0^{(1)} = x^{(1)})}{\P(X_0^{(1)} = x^{(1)})}\\
            &= \frac{\mu(x^{(1)}) \prod_{i=1}^{n-1} p(x_i^{(1)}, x_{i+1}^{(1)})}{\mu(x^{(1)})}\\
            &= \prod_{i=1}^{n-1} p(x_i^{(1)}, x_{i+1}^{(1)})
        \end{align*}
        Since, $p(x_i^{(1)}, x_{i+1}^{(1)})$ is a probability, $0 \leq p(x_i^{(1)}, x_{i+1}^{(1)}) \leq 1$. But by definition of the 1-dim SRW, 
        \[p(x, y) = \P(X_n = y \; | \; X_{n-1} = x) = \P(\xi_n = y - x) = \frac{1}{2} > 0\]
        So all the terms in the product are greater than $0$, and
        \[0 <  p(y^{(1)}, x^{(1)}) < \P(X_{n+1} = x_{n+1} \; | \; X_n = x_n)\]
        so the MC is irreducible. $\qed$
    \color{black}

\end{document}